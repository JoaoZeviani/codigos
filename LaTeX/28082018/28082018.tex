\documentclass[a4paper,12pt]{book}	
\usepackage[T1]{fontenc}
\usepackage[utf8]{inputenc}
\usepackage[brazilian]{babel}
%\usepackage[showframe]{geometry}
\usepackage{amsmath,amscd,amsthm,amssymb,amsxtra,latexsym}

\newtheorem{exemplo}{Exemplo}[chapter]

 \begin{document}

\setlength{\voffset}{-0.75in}
\chapter{}
\chapter{}
\chapter{}
\begin{exemplo}
asdasd
\end{exemplo}
\begin{exemplo}
asdasd
\end{exemplo}
\begin{exemplo}
asdasd
\end{exemplo}
\begin{exemplo}
asdasd
\end{exemplo}
\newpage
\thispagestyle{empty}
\hspace{-0.6cm}\textit{96}
\hfill
\textit{Introdução à Análise Combinatória}
\\\\
Se, a cada solução de $x_1+x_2+x_3=20$ em inteiros positivos com $x_2 > 5$, subtrairmos 5 unidades de $x_2$, teremos uma solução, em inteiros positivos, para $y_1+y_2+y_3=15$, onde
$$
y_1=x_1,\quad y_2=x_2-5 \; \mbox{ e }\; y_3=x_3
$$ 
Como a transformação acima é biunívoca e o número de soluções inteiras positivas de $y_1+y_2+y_3=15$ é $C_{14}^2$, este é o número de soluções inteiras positivas de  $x_1+x_2+x_3=20$, com $x_2>5. _\blacksquare$
\begin{exemplo}
Encontrar o número de soluções em inteiros positivos para a inequação
$$0<x_1+x_2+x_3+x_4 \leq 6$$
\end{exemplo}

Devemos contar o número de soluções em inteiros positivos para as seguintes equações:
\begin{eqnarray}
x_1+x_2+x_3+x_4&=&1;\nonumber \\
x_1+x_2+x_3+x_4&=&2;\nonumber \\
x_1+x_2+x_3+x_4&=&3;\nonumber \\
x_1+x_2+x_3+x_4&=&4;\nonumber \\
x_1+x_2+x_3+x_4&=&5;\nonumber \\
x_1+x_2+x_3+x_4&=&6 \nonumber
\end{eqnarray}
Como $C_0^3=C_1^3=C_2^3=0$, o número de soluções em inteiros positivos para cada uma das três primeiras equações é zero. Para as três últimas temos, respectivamente, $C_3^3=1, C_4^3=4, C_5^3=10.$ Logo, pelo princípio aditivo, o número procurado é $1+4+100=15._\blacksquare$
\begin{exemplo}
Encontrar o número de soluções em inteiros não-negativos de $x_1+x_2+x_3+x_4+x_5=18$, nas quais exatamente 2 incógnitas são nulas.
\end{exemplo}




\end{document}