\documentclass[a4paper,11pt]{article}	
\usepackage[T1]{fontenc}
\usepackage[utf8]{inputenc}
\usepackage[brazilian]{babel}
\usepackage{amsmath,amscd,amsthm,amssymb,amsxtra,latexsym}
\usepackage{xcolor}
\usepackage[all]{xy}

\newtheorem{teorema}{Teorema}[section]
\newtheorem{lema}[teorema]{Lema}
\newtheorem{prop}[teorema]{Proposição}
\newtheorem{corolario}[teorema]{Corolário}
\newtheorem{conjectura}{Conjectura}
\theoremstyle{definition}
\newtheorem{definicao}{Definição}[section]
\newtheorem{obs}{Observação}
\newtheorem{exemplo}{Exemplo}

\title{MA750 - Prova 1}
\author{Nome: João Vitor Zeviani	 \ \ RA: 199947}
\date{26/09/2018}

\begin{document}
\begin{table}[htpb]
\maketitle
\renewcommand\contentsname{Conteúdo}
\tableofcontents
\ 

\hspace{.8cm} Começamos nossa prova, gostaria de lembrar que:

\begin{description}
\item[i)] tem que ficar tranquilos, respira fundo e pense que você sabe resolver esta prova
{\color{red} \item[ii)] é extremamente proibido copiar de um companheiro. Pessoas que foram descobertas intercambiar informação, com qualquer meio, vão ter nota zero. Não podemos queixar-nos de um mundo desonesto se não damos o exemplo.}
\end{description}
\end{table}
\vspace{-0.8cm}
\section{O produto tensorial}
Podemos encontrar mais detalhes em  \cite{SLang}. Sejam $U$ e $W$ espaços vetorias de dimensão $n$ sobre um corpo $\mathbb{K}$
\begin{definicao}
Diremos que um espaço  vetorial denotado por $V \otimes W$, junto com uma aplicação bilinear

\begin{equation}
\begin{array}{cccc}
\alpha:& V \times W & \longrightarrow & V \otimes W\\
& (v,w) & \mapsto & v \otimes w
\end{array}\nonumber
\end{equation}
é o produto tensorial de $V$ e $W$ se, ao considerarmos um outro espaço vetorial $U$ sobre o mesmo corpo $\mathbb{K}$ e $\beta$ também uma aplicação bilinear:
\begin{equation}\label{beta}
\begin{array}{cccc}
\alpha:& V \times W & \longrightarrow & V \otimes W\\
& (v,w) & \mapsto & \beta (v,w) 
\end{array}
\end{equation}
afirmamos que existe uma, e somente uma, transformação \begin{em} linear \end{em}
$$ 
\mathcal{L}: V \otimes W \rightarrow U
$$

tal que $\beta(u,w) = \mathcal{L}(u\otimes w)$, isto é, $\beta = \mathcal{L}\circ \alpha$:

\begin{equation}\label{diagram}
\begin{gathered}
\xymatrix{
&V \otimes W \ar[rd]^{\mathcal{L}} & \\
V \times W \ar[ur]^{\alpha}  \ar[rr]_{\beta}&&U
}
\end{gathered}
\end{equation}

Ou seja, dizemos que a aplicação $\alpha$ é uma aplicação bilinear universal, pois qualquer outra é uma composição de $\alpha$ seguida por uma linear, vejam o diagrama (\ref{diagram}). Esta é a propriedade universal do produto tensorial, também chamada \begin{em} mapeamento universal \end{em} ou \begin{em} universalidade \end{em}.\\
Mais geralmente, o produto tensorial de $n$ espaços vetorias $ V_1, \dotsc , V_n$ é dado por um espaço vetorial que denotamos por
$$
V_1 \otimes \dotsc \otimes V_n,
$$
junto com uma aplicação $n$-linear universal:
$$
\alpha: V_1 \times \dotsc \times V_n \longrightarrow V_1 \otimes \dotsc \otimes V_n.
$$
\end{definicao}

\section{Concluímos}
Ânimo! Já teminou a prova (ou quasi)


\begin{quotation}
 {\itshape
Todas as cartas de amor são\\
Ridículas.\\
Não seriam cartas de amor de não fossem\\
Ridículas.

\vspace{.4cm}
Também escrevi em meu tempo cartas de amor,\\
Como as outras\\
Ridículas.

\vspace{.4cm}
As cartas de amor, se há amor,\\
Têm de ser\\
Ridículas.} \footnote{Fernando Pessoa, parte de  \textbf{ \textit {Todas as Cartas de Amor são Ridículas}}}
\end{quotation}
\begin{thebibliography}{9}
\bibitem{SLang} S. Lang \emph{Linear Algebra} Springer, 2013.
\end{thebibliography}
\addcontentsline{toc}{section}{Referências}
\end{document}