\documentclass[a4paper,12pt]{book}	
\usepackage[T1]{fontenc}
\usepackage[utf8]{inputenc}
\usepackage[brazilian]{babel}
\usepackage{enumerate}
\usepackage{graphicx, wrapfig}
\usepackage[leftcaption]{sidecap}
%\usepackage[showframe]{geometry}
\usepackage{amsmath,amscd,amsthm,amssymb,amsxtra,latexsym}
\newtheorem{exemplo}{Exemplo}[chapter]
\title{Introdução à Análise Combinatória}

\begin{document}
\maketitle
\setlength{\voffset}{-0.75in}

\chapter{}
\chapter{}
\chapter{}
\chapter{}
\chapter{Funções geradoras}
\begin{exemplo}
asdasd
\end{exemplo}
\begin{exemplo}
asdasd
\end{exemplo}
\begin{exemplo}
asdasd
\end{exemplo}
\newpage
a
\newpage
\thispagestyle{empty}
\hspace{-0.6cm}\textit{156}
\hfill
\textit{Introdução à Análise Combinatória}
\\\\
mas sempre que estivermos pedindo a função geradora (ordinária) estaremos interessados numa expressão simples (que comumente chamamos de "forma fechada") para a resposta. Neste caso, é fácil ver que
\begin{eqnarray}
f(x) & = & x^2+x^3+x^4+x^5+ \cdots \nonumber \\
& = &x^2(1+x+x^2+x^3+x^4+ \cdots ) \nonumber \\
& = & x^2\left(\frac{1}{1-x}\right).\blacksquare \nonumber 
\end{eqnarray}
\begin{exemplo}
Encontrar a sequência cuja função geradora é dada por:
$$g(x)=\frac{1}{1-x^2}.$$
\end{exemplo}
Sabemos que 
$$\frac{1}{1-x}=1+x+x^2+x^3+x^4+ \cdots .$$
Logo, basta substituirmos $x$ por $x^2$ nesta última expressão, obtendo:
$$\frac{1}{1-x^2}=1+x^2+x^4+x^6+x^8+ \cdots .$$
Portanto $g(x)$ é a fução geradora da sequência $(a_r)=(1,0,1,0,1,0,1,\dots).\blacksquare$
\begin{exemplo}\label{func:ger}
Encontrar a função geradora para a sequência
$$(a_r)=\left(1,\frac{1}{1!},\frac{1}{2!},\frac{1}{3!},\frac{1}{4!},\dots \right).$$
\end{exemplo}
Sabemos que a exámção em série de potência da função exponencial é igual a:
$$e^x=1+x+\frac{x^2}{2!}+\frac{x^3}{3!}+\frac{x^4}{4!}+\cdots +\frac{x^r}{r!}+\cdots .$$
Logo, a função procurada é $e^x.\blacksquare$
\newpage
\thispagestyle{empty}
\hspace{-0.6cm}\textit{Capítulo 5. Funções geradoras}
\hfill
\textit{157}
\\
\begin{exemplo}
Encontrar a sequência cuja função geradora ordinária é $ x^2+x^3+e^x$.
\end{exemplo}
Como
\begin{eqnarray}
x^2+x^3+e^x & = & x^2+x^3+\left(1+x+\frac{x^2}{2!}+\frac{x^3}{3!}+\frac{x^4}{4!}+\cdots \right) \nonumber \\
& = & 1+x+\left(1+ \frac{1}{2!} \right)x^2+\left(1+ \frac{1}{3!} \right)x^3+\frac{x^4}{4!}+ \cdots , \nonumber 
\end{eqnarray}
a sequência gerada por essa função é:
$$(a_r) = \left( 1,1,1+\frac{1}{2!},1+\frac{1}{3!},\frac{1}{4!},\frac{1}{5!},\cdots,\frac{1}{r!},\cdots \right).\blacksquare$$
\begin{exemplo}
Encontrar a função geradora ordinária para a sequência
\end{exemplo}
$$ (a_r) =\left(\frac{2^r}{r!}\right).$$
Observando-se os coeficientes de $x^r$ em 
$$e^x=1+x+\frac{x^2}{2!}+\frac{x^3}{3!}+\frac{x^4}{4!}+\cdots +\frac{x^r}{r!}+\cdots ,$$
é fácil ver que a substituição de $x$ por $2x$, isto é, calculando-se $e^2x$, teremos
\begin{eqnarray}
e^{2x} &=& 1+2x+\frac{{(2x)}^2}{2!}+\frac{{(2x)}^3}{3!}+\frac{{(2x)}^4}{4!}+\cdots +\frac{{(2x)}^r}{r!}+\cdots \nonumber \\
&=& 1+ \left( \frac {2^1}{1!} \right)x+\left( \frac {2^2}{2!} \right)x^2+\left( \frac {2^3}{3!} \right)x^3+\cdots +\left( \frac {2^r}{r!} \right)x^r+ \cdots , \nonumber
\end{eqnarray}
o que nos mostra ser $e^{2x}$ a função geradora procurada. $\blacksquare$
\begin{exemplo}
Qual o coeficiênte de $x^{23}$ na expanção de ${(1+x^5+x^9)}^6$?
\end{exemplo}
Soma de cincos e noves totalizando 23 só pode ser obtida somando-se 2 noves e um 5, isto é, $5+9+9$. Como são 6 fatores iguais a $(1+x^5+x^9)$, devemos escolher dois fatores, tomando $x^9$ em ambos, e

Como vimos no exmeplo \ref{func:ger}

\end{document}